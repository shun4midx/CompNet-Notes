\documentclass[12pt,a4paper]{article}
\input{shun4imports}
\input{shun4colors}
\newcommand{\bs}{\textbackslash}
\newcommand{\boxtxt}[1]{\boxed{\text{#1}}}

\tcbuselibrary{listings}
\newtcblisting{shuncode}{
    listing only,
    listing options={
        language=C,
        basicstyle=\ttfamily\small,
        keywordstyle=\color{blue},
        commentstyle=\color{green!50!black},
        stringstyle=\color{red}
    },
    colback=cyan!10,
    colframe=cyan!80!black,
    boxrule=2pt,
    left=0pt,
    right=0pt,
    top=0pt,
    bottom=0pt,
    boxsep=0pt,
    arc=5pt,
    enhanced
}

\lhead{Computer Networks}
\chead{Chapter 3 Summary}
\rhead{Shun (@shun4midx)}

\begin{document}
\begin{center}
  {\Large \bf Computer Networks: Chapter 3 Summary (MIDTERM VERSION)}\\[8pt]
  \textbf{Author:} Shun (@shun4midx)
\end{center}

\bluesec{Transport-Layer Services}

\noindent Transport services provide logical communication between application processes running on different hosts. Two transport protocols available to internet applications are \textbf{TCP and UDP}. \\

\pinkbox{Overview}{
  In transport protocols, the \textbf{sender breaks application messages into segments} and passes it to the \textbf{nextwork layer}, whereas the \textbf{receiver reassembles segments into messages}, and passes it into the \textbf{application layer}.
}

\pinkbox{Transport Layer vs Network Layer}{
  The \textbf{transport layer} is about the \textbf{communication between processes}, whereas the \textbf{network layer} is about the \textbf{communication between hosts}.
}  

\vspace{1.0em}\bluesec{Multiplexing and Demultiplexing}

\noindent To allow multiple applications to use the network simultaneously, the transport layer performs both \textbf{multiplexing} and \textbf{demultiplexing}. \\

\pinkbox{Definitions}{
  \begin{itemize}
    \item \textbf{Multiplexing (Sender side):} Gathering data from multiple sockets, adding transport headers (with port numbers and addresses), and passing the resulting segments to the network layer. \hlbf{Small to large transport.}
    \item \textbf{Demultiplexing (Receiver side):} Using header information (source/destination IP and port numbers) to deliver received segments to the correct socket/application process. \hlbf{Large to small transport.}
  \end{itemize}
}

\pinkbox{Header Information}{
  Each transport segment includes:

  \begin{itemize}
    \item Source \textbf{IP address}
    \item Destination \textbf{IP address}
    \item Source \textbf{port number}
    \item Destination \textbf{port number}
  \end{itemize}

  These fields are used by the receiver to identify the \textbf{appropriate receiving socket}.
}

\pinkbox{Connectionless vs Connection-Oriented Demultiplexing}{
  \begin{itemize}
    \item \textbf{UDP (Connectionless):} Each segment is directed to a socket based only on its \textbf{destination port number}.
  \hl{Multiple senders sending to the same port reach the same receiving socket.}
    \item \textbf{TCP (Connection-Oriented):} Each connection is identified by a unique \textbf{4-tuple}:
    \[
    \hl{(\text{Source IP}, \text{Source Port}, \text{Destination IP}, \text{Destination Port})}
    \]
    A server can distinguish multiple TCP connections on the same port (e.g., port 80 for multiple clients).
  \end{itemize}
}

\vspace{1.0em}\bluesec{[Safe for Exam Criteria] UDP Checksum (Safety Addition)}

\noindent The \textbf{UDP checksum} provides simple error detection. \hl{Its purpose is to detect bit errors, not correct them.} \\

\pinkbox{Overview}{
  \noindent It is computed by treating the UDP segment as a sequence of 16-bit integers, summing them using \textbf{one’s-complement arithmetic}, and taking the \textbf{one’s complement} of the result.

  \noindent At the receiver, all words (including the checksum) are summed again---if the result is all 1s, no error is detected.
}

\end{document}